% \documentclass{ctexart}
\documentclass{article}
\usepackage{listings}                                           %插入代码
\usepackage{geometry}                                           %设置页面大小边距等
\usepackage{graphicx}                                           %插入图片
\usepackage{amssymb}                                            %为了用\mathbb
\usepackage{amsmath}                                            %数学方程的显示
\usepackage{listings}                                           %插入代码
\usepackage{fancyhdr}                                           %设置页眉页脚
\usepackage{lastpage}                                           %总页数
\usepackage{hyperref}                                           %引用网页
\usepackage{xcolor}
\usepackage{tikz}
\usepackage{float}
\usepackage{subcaption} 
\usepackage{mathrsfs}
\usepackage{amsthm}



\geometry{a4paper,left=2cm,right=2cm,top=2cm,bottom=2cm}        %一定要放在前面!
\pagestyle{fancy}                                               %设置页眉页脚
\lhead{Guanyu Chen}                                             %页眉左Fir                                        

\rhead{Stochastic Differential Equation} 
\cfoot{\thepage/\pageref{LastPage}}                             %当前页,记得调用前文提到的宏包
\lfoot{Zhejiang University}
\rfoot{College Of Integrated Circuits}
\renewcommand{\headrulewidth}{0.1mm}                            %页眉线宽,设为0可以去页眉线
\renewcommand{\footrulewidth}{0.1mm}                            %页脚线宽,设为0可以去页眉线
\setlength{\headwidth}{\textwidth}

\hypersetup{                                                    %设置网页链接颜色等
    colorlinks=true,                                            %链接将会有颜色,默认是红色
    linkcolor=blue,                                             %内部链接,那些由交叉引用生成的链接将会变为蓝色(blue)
    filecolor=magenta,                                          %链接到本地文件的链接将会变为洋红色(magenta)
    urlcolor=blue,                                              %链接到网站的链接将会变为蓝绿色(cyan)
}

\lstset{  
    basicstyle=\ttfamily,  
    keywordstyle=\color{blue},  
    language=Python,  
    numbers=left,  
    numberstyle=\tiny\color{gray},  
    frame=single,  
    breaklines=true  
}  

\newtheorem{theorem}{Theorem}
% \newtheorem{proof}{Proof}
\newtheorem{solution}{Solution:}
\newtheorem{remark}{Remark}
\newtheorem{definition}{Definition}
\newtheorem{algorithm}{Algorithm}
\newtheorem{lemma}{Lemma}
\newtheorem{example}{Example}
\newtheorem{problem}{Problem}

\title{About Stochastic Differential Equation}
\author{Guanyu Chen}
\date{\today}
\begin{document}
\maketitle
\tableofcontents
\newpage
\section{Fokker-Planck-Kolmogorov equation}
\begin{problem}
Assume we have a Stochastic Differential Equation like:
\begin{equation}\label{sde}
    dX_t = f(X_t, t)dt + G(X_t, t)dW_t
\end{equation}
where $X_t\in \mathbf{R}^d,f\in \mathcal{L}(\mathbf{R}^{d+1}, \mathbf{R}^d)$, and $W_t$ is m-dim Brownian Motion with diffusion matrix $Q$, 
$G(X_t, t)\in \mathcal{L}(\mathbf{R}^{m+1}, \mathbf{R}^d)$, with initial condition $X_0\sim p(X_0)$.
\end{problem}
\begin{definition}[Generator]
    The infinitesimal generator of a stochastic process $X(t)$ for function $\phi(x)$, i.e. $\phi(X_t)$ can be defined as
    \begin{equation}
        \mathcal{A} \phi(X_t)=\lim _{s \rightarrow 0^{+}} \frac{E[\phi(X(t+s)]-\phi(X(t))}{s}
    \end{equation}
    Where  $\phi$  is a suitable regular function.
\end{definition}
This leads to Dynkin's Formula very naturally.
\begin{theorem}[Dynkin's Formula]
    \begin{equation}
        E[f(X_t)]=f(X_0)+E\left[\int_0^t\mathcal{A}(f(X_s))ds\right]
    \end{equation}
\end{theorem}

\begin{theorem}
    If  $X(t)$  s.t. \ref{sde}, then the generator is given:
\begin{equation}
    \mathcal{A}(\cdot)=\sum_{i} \frac{\partial(\cdot)}{\partial x_{i}} f_{i}(X_t, t)+\frac{1}{2} \sum_{i, j}\left(\frac{\partial^{2}(\cdot)}{\partial x_{i} \partial x_{j}}\right)\left[G(X_t, t)Q G^{\top}(X_t, t)\right]_{i j}
\end{equation}
\end{theorem}
\begin{proof}
    See P119 of SDE by Oksendal.
\end{proof}

\begin{example}
    If $dX_t=dW_t$, then $\mathcal{A}=\frac{1}{2}\Delta$, where $\Delta$ is the Laplace operator.
\end{example}

\begin{definition}[Generalized Generator]
    For $\phi(x, t)$, i.e. $\phi(X_t, t)$, the generator can be defined as:
    \begin{equation}
        A_{t} \phi(x, t)=\lim _{s \rightarrow 0^{+}} \frac{E[\phi(X(t+s), t+s)]-\phi(X(t), t)}{s}
    \end{equation}
\end{definition}

\begin{theorem}
    Similarly if $X(t)$ s.t. \ref{sde}, then the generalized generator is given:
    \begin{equation}
        \mathcal{A}_{t}(\cdot)=\frac{\partial(\cdot)}{\partial t}+\sum_{i} \frac{\partial(\cdot)}{\partial x_{i}} f_{i}(X_t, t)+\frac{1}{2} \sum_{i, j}\left(\frac{\partial^{2}(\cdot)}{\partial x_{i} \partial x_{j}}\right)\left[G(X_t, t) Q G^{\top}(X_t, t)\right]_{i j}
    \end{equation}
\end{theorem}
We want to consider the density distribution of $X_t, P(x, t)$
\begin{theorem}[Fokken-Planck-Kolmogorov equation]
    The density function $P(x, t)$ of $X_t$ s.t. \ref{sde} solves the PDE:
    \begin{equation}
        \frac{\partial P(x, t)}{\partial t}=-\sum_{i} \frac{\partial}{\partial x_{i}}\left[f_{i}(x, t) p(x, t)\right]+\frac{1}{2} \sum_{i, j} \frac{\partial^{2}}{\partial x_{i} \partial x_{j}}\left[\left(G Q G^{\top}\right)_{i j} P(x, t)\right]
    \end{equation}
    The PDE is called FPK equation / forwand Kolmogorov equation.
\end{theorem}
\begin{proof}
    Consider the function $\phi(x)$, let $x=X_t$ and apply Ito's Formula:
    \begin{equation}
        \begin{aligned}
            d \phi & =\sum_{i} \frac{\partial \phi}{\partial x_{i}} d x_{i}+\frac{1}{2} \sum_{i, j}\left(\frac{\partial^{2} \phi}{\partial x_{i} \partial x_{j}}\right) d x_{i} d x_{j} \\
            & =\sum_{i} \frac{\partial \phi}{\partial x_{i}}\left(f_{i}\left(X_t, t\right) d t+\left(G\left(X_{t}, t\right) d W_{t}\right)\right)+\frac{1}{2} \sum_{i, j}\left(\frac{\partial^{2} \phi}{\partial x_{i} \partial x_{j}}\right)\left[G(X_t, t) Q G^{\top}(X_t, t)\right]_{i j} d t .
            \end{aligned}
    \end{equation}
    Take expectation of both sides:
    \begin{equation}\label{expectation}
        \frac{d E[\phi]}{d t}=\sum_{i} E\left[\frac{\partial \phi}{\partial x_{i}} f_{i}(X_t, t)\right]+\frac{1}{2} \sum_{i j} E\left[\frac{\partial^{2} \phi}{\partial x_{i} \partial x_{j}}\left[G Q G^{\top}\right]_{i j}\right]
    \end{equation}
    So 
    \begin{equation}\left\{
        \begin{aligned}
            &\frac{d E[\phi]}{d t} =\frac{d}{d t}\left[\int \phi(x) P(X_t=x, t) d x\right]=\int \phi(x) \frac{\partial P(x, t)}{\partial t} dx\\
            &\sum_{i} E\left[\frac{\partial \phi}{\partial x_{i}} f_{i}\right]=\sum_{i} \int\frac{\partial \phi}{\partial x_{i}} f_{i}(X_t=x, t) P d x
            =-\sum_{i} \int \phi \cdot \frac{\partial}{\partial x_{i}}\left[f_{i}(x, t) p(x, t)\right] d x . \\
            &\frac{1}{2} \sum_{i j} E\left[\frac{\partial^{2} \phi}{\partial x_{i} \partial x_{j}}\left[G Q G^{\top}\right]_{i j}\right]=\frac{1}{2} \sum_{i j} \int \frac{\partial^{2} \phi}{\partial x_{i} \partial x_{j}}\left[G Q G^{\top}\right]_{i j} P d x
            =\frac{1}{2} \sum_{i j} \int \phi(x) \frac{\partial^{2}}{\partial x_{i} \partial x_{j}}\left(\left[G Q G^{\top}\right]_{i j} P\right) d x. \\
        \end{aligned}\right.
    \end{equation}
    then
    $$\int \phi  \frac{\partial P}{\partial t} d X=-\sum_{i} \int \phi  \frac{\partial}{\partial x_{i}}\left(f_{i} P\right) d X+\frac{1}{2} \sum_{i j} \int \phi \frac{\partial^{2}}{\partial x_{i} x_{j}}\left(\left[G Q G^{\top}\right]_{i j} P\right) d x$$
    Hence $$\int \phi \cdot\left[\frac{\partial P}{\partial t}+\sum_{i} \frac{\partial}{\partial x_{i}}\left(f_{i} P\right)-\frac{1}{2} \sum_{i j} \frac{\partial^{2}}{\partial x_{i} \partial x_{j}}\left(\left[G Q G^{\top}\right]_{i j} P\right)\right] d X=0$$
    Therefore P s.t.    
    \begin{equation}
        \frac{\partial P}{\partial t}+\sum_{i} \frac{\partial}{\partial x_{i}}\left(f_{i}(x, t) P(x, t)\right)-\frac{1}{2} \sum_{i=1} \frac{\partial^{2}}{\partial X_{i} \partial X_{j}}\left(\left[G Q G^{\top}\right]_{i j} P\left(x,t\right)\right)=0
    \end{equation}
    Which gives the FPK Equation.
\end{proof}

\begin{remark}
    When SDE is time independent:  
    \begin{equation}
        d X_t=f(X_t) d t+G(X_t) d W_{t}  
    \end{equation}
    then the solution of FPK often converges to a stationary solution s.t.  $\frac{\partial P}{\partial t}=0$.
\end{remark}
Here is an another way to show FPK equation: Since we have inner product $\langle\phi, \psi\rangle=\int \phi(x)\psi(x)dx$. Then $E[\phi(x)]=\langle\phi, P\rangle$.

As the equation \ref{expectation} can be written as 
\begin{equation}
    \frac{d}{dt}\langle\phi, P\rangle=\langle\mathcal{A}\phi, P\rangle
\end{equation}
Where $\mathcal{A}$ has been mentioned above. If we note the adjoint operator of $\mathcal{A}$ as $\mathcal{A}^*$, then we have
\begin{equation}
    \langle\phi, \frac{dP}{dt}-\mathcal{A}^*(P)\rangle=0,\forall \phi(x)
\end{equation}
Hence we have 
\begin{theorem}[FPK Equation]
    \begin{equation}
    \frac{dP}{dt}=\mathcal{A}^*(P),\operatorname{where} \mathcal{A}^*(\cdot)=-\sum_{i} \frac{\partial}{\partial x_{i}}\left(f_{i}(x, t) (\cdot)\right)+\frac{1}{2} \sum_{i=1} \frac{\partial^{2}}{\partial X_{i} \partial X_{j}}\left(\left[G Q G^{\top}\right]_{i j}(\cdot)\right)
\end{equation}
\end{theorem}

\begin{theorem}[Transition Density(Forward Komogorov Equation)]
     The transition density $P_{t|s}(x_t|x_s),t\geq s$, which means the propability of transition from $X(s)=x_s$ to $X(t)=x_t$, satisfies the FPK equation with initial condition $P_{s|s}(x|x_s)=\delta(x-x_s)$
     i.e. for $P_{t|s}(x|y)$, it solves
     \begin{equation}
        \frac{\partial P_{t|s}(x|y)}{\partial t}=\mathcal{A}^*(P_{t|s}(x|y)), \operatorname{with} P_{s|s}(x|y)=\delta(x-y)
     \end{equation}
\end{theorem}

\begin{theorem}[Backward Komogorov Equation]
    $P_{s|t}(y|x)$ for $t\geq s$ solves:
    \begin{equation}
        \frac{\partial P_{s|t}(y|x)}{\partial s} + \mathcal{A}(P_{s|t}(y|x))=0, \operatorname{ with }P_{s|t}(y|x) = \delta(x-y)
    \end{equation}
\end{theorem}


\section{Feynman-Kac Formula}
The Feynman-Kac Formula bridges PDE and certain stochastic value of SDE solutions.

Consider $u(x, t)$ satisfied the following PDE:
\begin{equation}
    \frac{\partial u}{\partial t}+f(x) \frac{\partial u}{\partial x}+\frac{1}{2} L^{2}(x) \frac{\partial^{2} u}{\partial x^{2}}=0 . \quad u(x, T)=\psi(x) .
\end{equation}

Then we define a stochastic process $X(t)$  on  $\left[t^{\prime}, T\right]$  as

\begin{equation}
    d X=f(X) d t+L(X) d W_{t} \quad X\left(t^{\prime}\right)=x^{\prime}
\end{equation}
By Ito formula:
\begin{equation}
    \begin{aligned}
    d u & =\frac{\partial u}{\partial t} d t+\frac{\partial u}{\partial x} d x+\frac{1}{2} \frac{\partial^{2} u}{\partial x^{2}} d x^{2} \\
    & =\frac{\partial u}{\partial t} d t+\frac{\partial u}{\partial x}\left(f(x) d t+L(x) d W_{t}\right)+\frac{1}{2} \frac{\partial^{2} u}{\partial x^{2}} L^{2}(x) d t \\
    & =\left(\frac{\partial u}{\partial t}+\frac{\partial u}{\partial x} f(x)+\frac{1}{2} \frac{\partial^{2} u}{\partial x^{2}} L^{2}(x)\right) d t+\frac{\partial u}{\partial x} L(x) d W_{t} . \\
    & =\frac{\partial u}{\partial x} L(x) d W_{t} .
    \end{aligned}
\end{equation}

Integrating both sises from $t^{\prime}$  to  T:
\begin{equation}
    \begin{aligned}
        \int_{t^{\prime}}^{T} \frac{\partial u}{\partial x} L(x) d W_{t} &= u(X(T), T) - u(X(t'), t')\\
        &= \psi(X(T)) - u(x', t')
    \end{aligned}
\end{equation}
Take expectation of both sides:
\begin{equation}
    u\left(x^{\prime}, t^{\prime}\right)=E[\Psi(X(T))]
\end{equation}  
This can be generalized to PDE like:
\begin{equation}
    \frac{\partial u}{\partial t}+f(x) \frac{\partial u}{\partial x}+\frac{1}{2} L^{2}(x) \frac{\partial^{2} u}{\partial x^{2}}-r u=0 . \quad u(x, T)=\psi(x) \text {. }
\end{equation}
By consider the Ito formula of $e^{-rt}u(x, t)$, we can similarly compute the resulting Feynman-Kac equation as 
\begin{equation}
    u(x', t') = e^{-r(T-t')}E\left[\psi(X(T))\right]
\end{equation}
This means we can get the value of PDE at $(x', t')$ by simulating SDE paths beginning at $(x', t')$, and compute corresponding $E\left[\psi(X(T))\right]$. We can get more generalized conclusion:
\begin{theorem}[Solve Backward PDE]

    To compute the backward PDE: $(\mathcal{A}_t-r)(u)=0$, i.e.
    \begin{equation}
        \frac{\partial u}{\partial t} + \sum_{i} \frac{\partial u(x, t)}{\partial x_{i}} f_{i}(x, t)+\frac{1}{2} \sum_{i, j}\left(\frac{\partial^{2}u(x, t)}{\partial x_{i} \partial x_{j}}\right)\left[G(x, t)Q G^{\top}(x, t)\right]_{i j} - ru(x, t)=0
    \end{equation}
    with boundary condition $u(x, T)=\psi(x)$. Then for any fixed points $(x', t')$ where $t'\leq T, x'\in D$, $u(x', t')$ can be computed as:\\
    Step1. Simulate N sample paths of SDE from $t'$ to $T$:
    \begin{equation}
        dX_t=f(X_t, t)dt + G(X_t, t)dW_t\operatorname{with}X(t')=x'
    \end{equation}
    Step2. Estimate $u(x', t') = e^{-r(T-t')}E\left[\psi(X(T))\right]$
\end{theorem}

\begin{theorem}[Solve Forward PDE]
    Consider the solution $u(x, t)$ of forward PDE: $\frac{\partial u}{\partial t}=(\mathcal{A}-r)(u)$, i.e.
    \begin{equation}
        \frac{\partial u}{\partial t}=\sum_{i} \frac{\partial u(x, t)}{\partial x_{i}} f_{i}(x, t)+\frac{1}{2} \sum_{i, j}\left(\frac{\partial^{2}u(x, t)}{\partial x_{i} \partial x_{j}}\right)\left[G(x, t) Q G^{\top}(x, t)\right]_{i j} - ru(x, t)
    \end{equation}
    with initial condition $u(x, 0)=\psi(x)$. Then for any fixed points $(x', t')$ where $t'\leq T, x'\in D$, $u(x', t')$ can be computed as:\\
    Step1. Simulate N sample paths of SDE from $0$ to $t'$:
    \begin{equation}
        dX_t=f(X_t, t)dt + G(X_t, t)dW_t\operatorname{with}X(0)=x'
    \end{equation}
    Step2. Estimate $u(x', t') = e^{-rt'}E\left[\psi(X(t'))\right]$
\end{theorem}
\begin{theorem}[Solve Boundary Value Problem]
    For solution $u(x)$ to the following elliptic PDE defined on some domain $D$:
    \begin{equation}
        \sum_{i} \frac{\partial u(x)}{\partial x_{i}} f_{i}(x)+\frac{1}{2} \sum_{i, j}\left(\frac{\partial^{2}u(x)}{\partial x_{i} \partial x_{j}}\right)\left[G(x) Q G^{\top}(x)\right]_{i j} - ru(x)=0
    \end{equation}
    with boundary condition $u(x)=\psi(x)$ on $\partial D$. Then for any fixed points in $D$ can be computed as:\\
    Step1. Simulate N sample paths of SDE from $t'$ to the first exit time $T_e$:
    \begin{equation}
        dX_t=f(X_t)dt + G(X_t)dW_t\operatorname{with}X(t')=x'
    \end{equation}
    Step2. Estimate $u(x') = e^{-r(T_e-t')}E\left[\psi(X(T_e))\right]$
\end{theorem}

\section{Linear Filtering Problem}
\section{Parameter Estimation in SDE}
\end{document} 